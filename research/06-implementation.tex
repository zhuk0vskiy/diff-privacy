\chapter{Сравнение}

В таблице приведена классификация ранее рассмотренныего ПО с дифференциальной приватности для баз данных.
По горизонтали расположены критерии сравнения методов функционала ПО.

\begin{enumerate}
	\item  тип приватности;
	\item поддерживаемые механизмы приватности;
	\item тип реализации;
\end{enumerate}

\begin{table}[h]
	\centering
	\begin{tabular}{|c|c|c|c|}
		\hline
		ПО & 1 & 2 & 3 \\ \hline
		IBM Diffprivlib & $\epsilon$, $(\epsilon$, $\delta)$  & \parbox{4.5cm}{Лапласа, Гаусса, Экспоненциальный} & интеграция  \\ \hline
		PINQ &  $\epsilon$ & Лапласа  & фреймворк  \\ \hline
		GoogleSQL  & $\epsilon$, $(\epsilon$, $\delta)$ & \parbox{4.5cm}{Лапласа, Гаусса, Экспоненциальный}  & уровень СУБД \\ \hline
		DP SQL  & $\epsilon$ & \parbox{4.5cm}{Лапласа, Гаусса}  & уровень СУБД \\ \hline
	\end{tabular}
	\caption{Пример таблицы}
\end{table}

Выявлено, что каждое ПО имеет ряд своих преимуществ и ограничений.
Библиотека IBM Diffprivlib имеет широкий функционал реализации ДП, благодаря поддержке большого спектра возможностей в виде механизмов приватности и типов приватности.
К особенности можно отнести работу с базами данных через стороннее ПО.
Данная библиотеку можно использовать в направлении Data Science.

Библиотека PINQ обладает меньшим функционалом, чем IBM Diffprivlib, но имеет встроенную возможность взаимодействия с БД, что упрощает использование.
Является самой популярной библиотекой для ДП на языке C\#, поэтому подойдет для использования в проектах на языке C\#.

GoogleSQL также обладает широким функционалом реализации ДП, но уже на уровне СУБД, что упрощает и усикоряет работу с конкретной БД.
Но GoogleSQL является проприетарным решением компании Google на платформе Google Cloud, поэтому данное ПО подойдет для коммерческих проектов на базе Google Cloud.

DP SQL обладает менее широким функционалом, чем GoogleSQL, но является свободным распростроняемым расширением для PostrgeSQL.
Поэтому данный вариант подойдет при работе с PostgreSQL


%ПО с многоуровневой реализацией, в отличии от монолитной реализации, предлагает более гибкое решение с возможностью использования любого промежуточного фреймворка для работы с базами данных.
%Также имеет влияние язык программирования, на котором было написано ПО с многоуровневой и монолитной реализацией, что накладывает ограничение на использование.

%При реализации на уровне СУБД явным преимуществом является возможность работы с базами данных без каких либо дополнительных вмешательств на стороне клиента.
%Так же поддерживаемые типы приватности прямо влияют на уровень конфиденциальности данных, но также и влияют на уровень шума, добавляемого в данные.
%Количество поддерживаемых механизмов влияет на функционал ПО, предосталяемого пользователю, т.~к. каждый из механизмов предназначен для обработки данных разного типа. 
