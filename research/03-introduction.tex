\chapter*{ВВЕДЕНИЕ}
\addcontentsline{toc}{chapter}{ВВЕДЕНИЕ}
В 2023 году ЭАЦ зарегистрировал 11549 утечек информации, произошедших в мире, что
на 61,5\% больше, чем в 2022 году, когда в различных источниках были найдены
сведения о 7149 утечках, как показано на рисунке~\ref{img:ved}. Впервые за многолетнюю историю зафиксировано пятизначное количество утечек данных.~\cite{utechki}.

Также, с ссылкой на источник~\cite{forbes}, объем слитых персональных данных в России в 2023 году составил 1,12 млрд записей. Всего за отчетный период из российских компаний утекло 95 крупных баз данных.
 
  \includeimage{ved}{f}{h}{0.8\textwidth}{Количество утечек}
 

Цель работы --- провести обзор существующего программного обеспечения с возможностью
дифференциальной приватности для баз данных.
Для достижения этой цели требуется решить следующие задачи:
\begin{enumerate}[label*=---]
	\item провести анализ предметно области; 
	\item провести обзор существующего ПО;
	\item определить критерии сравнения существующего ПО;
	\item классифицировать существующее ПО.
\end{enumerate}